{
\centering
\textbf{چکیده}

\vspace{\baselineskip}
\textbf{\PersianTitle}

\vspace{2\baselineskip}
به کوشش \\
\textbf{\PersianName} \par
}

\vspace{2\baselineskip}
مغز و سیستم عصبی به عنوان پیچیده‌ترین بخش از موجودات زنده، نقش بسیار مهمی در پردازش، ذخیره و انتقال اطلاعات دارند.
نورون‌ها که از اجزای اصلی این سیستم هستند، ممکن است دچار اختلال‌ها و بیماری‌های مختلفی شوند.
بسیاری از مشکلات و بیماری‌های عصبی ناشی از عملکرد نادرست نورون‌هاست.
بنابراین شناخت و کنترل عملکرد این سلول‌ها برای درمان بیماری‌های عصبی از اهمیت ویژه‌ای برخوردار است.
نورون‌ها سیستم‌هایی دینامیکی هستند و بیماری‌های مرتبط با آن‌ها را می‌توان به صورت تغییر در پارامترهای مدل‌ دینامیکی آن‌ها توصیف کرد.
برای بررسی تعداد جواب‌های معادله دینامیکی که فاصله مشخصی از دنباله مرجع (خروجی یک نورون سالم) دارند، از مفهوم آنتروپی استفاده می‌کنیم.
با استفاده از الگوریتم انتشار باور و محاسبه چگالی انرژی و چگالی آنتروپی، رفتار فاصله دنباله‌ی مشاهده شده را در یک مدل تک نورونی (نگاشت چیالوو) بررسی کردیم.
نتایج نشان دادند که در مرز ناحیه برانگیخته و همچنین مرز ناحیه آشوبناک، با کاهش فاصله از دنباله مرجع، تعداد جواب‌های معادله دینامیکی کاهش می‌یابد.
هرچند این کاهش در مرز ناحیه برانگیخته شیب بیشتری دارد.
علاوه بر این، با استفاده از روش مونته‌کارلو، خطای پارامترهای مدل تک نورونی را تصحیح کردیم.
مشاهده شد که در مرز ناحیه برانگیخته، خطای پارامترها با افزایش طول دنباله‌ی مشاهده شده به سرعت کاهش یافته و در طول مشخصی به کمینه مقدار خود می‌رسد و سپس با افزایش بیشتر طول، نسبتاً ثابت باقی می‌ماند.
در مرز ناحیه آشوبناک نیز ابتدا خطا با افزایش طول، نسبت به مرز ناحیه برانگیخته، به کندی کاهش می‌یابد و برخلاف آن، با افزایش بیشتر طول، مقدار خطا افزایش پیدا می‌کند.

\vspace{\baselineskip}
\textbf{واژگان کلیدی:}
انتشار باور، تصحیح خطا، مدل نورونی
