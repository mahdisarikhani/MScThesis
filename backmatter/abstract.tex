\begin{latin}
    \setlength{\baselineskip}{16pt}
    {

        \centering
        \textbf{Abstract}

        \vspace{3\baselineskip}
        \textbf{\LatinTitle}

        \vspace{2\baselineskip}
        By \\
        \textbf{\LatinName} \par
    }

    \vspace{\baselineskip}
    The brain and nervous system are the most complex parts of living beings and play crucial roles in processing, storing, and transmitting information.
    Neurons, the primary building blocks of this system, are susceptible to various disorders and diseases.
    Many neurological issues and diseases arise from the malfunctioning of neurons.
    Therefore, understanding and regulating the function of these cells is essential for treating neurological diseases.
    Neurons are dynamical systems, and diseases related to them can be described as changes in the parameters of their dynamical model.
    In this study, we employed the concept of entropy to evaluate the number of dynamical solutions that maintain a specific distance from a reference sequence (the output of a healthy neuron).
    Using the belief propagation algorithm along with calculations of the energy density and the entropy density, we investigated the behavior of the distance of the observed sequence in a single-neuron model (the Chialvo map).
    Results indicate that at the boundary of the excitable regime and the boundary of the chaotic regime, the number of dynamical solutions decreases as the distance from the reference sequence decreases.
    This decrease, however, is more pronounced at the boundary of the excitable regime.
    Furthermore, using the Monte Carlo method, we corrected the parameter errors in the single-neuron model.
    It was observed that at the boundary of the excitable regime, the parameter error decreases rapidly as the observed sequence increases, reaching a minimum value at a specific length and then remaining relatively constant with further increases.
    However, at the boundary of the chaotic regime, the error initially decreases more slowly than in the excitable regime and, in contrast, begins to increase as the sequence length continues to grow.

    \textbf{Keywords:} Belief propagation, Error correction, Neural model
\end{latin}
