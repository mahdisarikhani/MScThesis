\chapter{مقدمه}

همان‌طور که عملکرد هر سیستم طبیعی از تعامل اجزای سازنده آن نشأت می‌گیرد، عملکرد مغز نیز نتیجه تعامل بین اجزای مختلف آن است.
مغز، به عنوان یک سیستم پیچیده، از حدود ۸۶ میلیارد نورون تشکیل شده است که در یک ساختار شبکه‌ای به هم متصل شده‌اند
\cite{herculano-houzel2009}.
این شبکه به‌گونه‌ای است که هر نورون به طور معمول به بیش از ۱۰ هزار نورون دیگر متصل است و از آن‌ها ورودی دریافت می‌کند
\cite{izhikevich2006,trappenberg2022}.
هرچند مغز شامل سلول‌های دیگری نیز هست، اما وظیفه اصلی پردازش، ذخیره و انتقال اطلاعات بر عهده نورون‌هاست و دیگر سلول‌ها بیشتر نقش حمایتی از نورون را دارند.

با توجه به ماهیت بیولوژیک مغز و نورون‌ها، این سیستم همواره در معرض اختلال و بیماری‌های مختلف قرار دارد.
بسیاری از مشکلات و بیماری‌های مغز و سیستم عصبی را می‌توان به عملکرد نادرست نورون‌ها نسبت داد.
از این رو، شناخت و کنترل نورون‌ها نقشی اساسی در درمان برخی از بیماری‌های مرتبط با سیستم عصبی دارد.
عوامل مختلفی مانند تغییر در چگالی یون‌ها و پروتئین‌های اطراف نورون، اتصال پروتئین‌های مختلف به غشای نورون و تغییر در رفتار کانال‌های یونی می‌توانند منجر به اختلال در رفتار دینامیکی نورون شوند.
این اختلال‌های بیولوژیکی در نورون را می‌توان به شکل اختلال‌هایی در پارامترهای یک مدل دینامیکی از نورون تلقی کرد.
همچنین، می‌توان درمان بیماری‌های نورون را به شکل تصحیح این اختلال‌ها در پارامترهای یک مدل دینامیکی از نورون در نظر گرفت
\cite{shaw2017}.

هدف از مطالعه سیستم‌های دینامیکی، نه تنها شناخت آن‌ها، بلکه به دست آوردن توانایی در پیش‌بینی و کنترل رفتار آن‌هاست.
از دیدگاه سیستم‌های دینامیکی، کنترل‌پذیری به معنای توانایی رساندن سیستم از هر حالت اولیه به هر حالت نهایی دلخواه در زمانی محدود است.
به طور کلی، کنترل و تصحیح یک سیستم مستلزم دو پیش‌نیاز است؛
شناخت دقیق قوانین دینامیکی حاکم بر رفتار سیستم و توانایی تأثیرگذاری بر رفتار زمانی تعداد مشخصی از پارامترهای سیستم.
بازخورد کلید اصلی کنترل یک سیستم است.
تفاوت بین مقدار واقعی و مقدار مطلوب به عنوان بازخورد به سیستم داده می‌شود تا آن را به حالت نهایی مطلوب هدایت کند
\cite{liu2016,boccaletti2000}.

نورون‌ها، سیستم‌هایی دینامیکی هستند.
یک سیستم دینامیکی از مجموعه‌ای از متغیرها تشکیل شده است که حالت آن سیستم را توصیف می‌کند.
همچنین، قانونی بر این متغیرها حاکم است که تحول آن‌ها را در طول زمان در دست دارد
\cite{izhikevich2006}.

نورون‌ها به عنوان واحدهای پردازش اولیه در سیستم عصبی در یک الگوی پیچیده به یکدیگر متصل هستند
\cite{gerstner2002}.
این سلول‌ها وظیفه انتقال و محاسبه اطلاعات را بر عهده دارند و مهم‌ترین جزء از اجزای تشکیل دهنده‌ی سیستم عصبی محسوب می‌شوند
\cite{graben2008}.
ویژگی منحصربه‌فرد نورون‌ها، انتشار سریع سیگنال‌های الکتریکی در فواصل زیاد است
\cite{dayan2001}.

ورودی‌هایی الکتریکی و شیمیایی که یک نورون از دیگر نورون‌ها و محیط اطراف دریافت می‌کند، منجر به ایجاد یک جریان الکتریکی در آن نورون می‌شود.
این جریان الکتریکی، تغییراتی در پتانسیل غشای نورون به وجود می‌آورد.
این تغییرات که به پتانسیل پس‌سیناپسی\LTRfootnote{Postsynaptic potential} معروف هستند، بسته به شدت جریان، می‌توانند کوچک یا بزرگ باشند.
پتانسیل پس‌سیناپسی قوی می‌تواند کانال‌های حساس به پتانسیل الکتریکی در غشای نورون را تحریک کرده و منجر به تولید پتانسیل عمل\LTRfootnote{Action potential} یا اسپایک\LTRfootnote{Spike} شود.
پتانسیل عمل، یک تغییر ناگهانی و گذرا در پتانسیل غشای نورون است که از طریق آکسون به دیگر نورون‌ها منتقل می‌شود.
نورون‌ها به شکل خودبه‌خودی شلیک نمی‌کنند، بلکه این شلیک‌ها نتیجه دریافت ورودی از سایر نورون‌ها هستند
\cite{izhikevich2006}.

غشای نورون پوشیده از کانال‌های یونی است و خود نورون نیز در یک محیط پلاسمایی غوطه‌ور است.
چگالی یون‌هایی مانند سدیم، پتاسیم، کلسیم و کلرید در داخل و خارج نورون بسیار متفاوت است.
این تفاوت چگالی باعث وارد شدن نیرویی قابل توجه به دیواره و غشای نورون می‌شود.
این نیرو از دو نیروی متضاد هم تشکیل شده است.
نیروی اسمزی که ناشی از تفاوت چگالی در دو سوی غشای نورون است و نیروی کولنی که به دلیل تفاوت در پتانسیل الکتریکی دو سوی غشاء ایجاد می‌شود.

هدف اصلی مدل‌های دینامیکی نورون، بازتولید حالت‌های فیزیولوژیکی نورون‌ها در جهت درک و پیش‌بینی رفتار آن‌ها در پاسخ به اختلال‌های مختلف است
\cite{chialvo1995}.
مدل‌های متعددی برای توصیف ساختار و رفتار دینامیکی نورون‌ها ارائه شده است، اما یکی از دقیق‌ترین مدل‌ها را آلن هاجکین\LTRfootnote{Alan Hodgkin} و اندرو هاکسلی\LTRfootnote{Andrew Huxley} در سال ۱۹۵۲ با معرفی معادلاتی که به مدل هاجکین-هاکسلی معروف است، ارائه کردند
\cite{hodgkin1952}.

این مدل با وجود دقت بالا در توصیف رفتار نورون، از نظر محاسباتی بسیار پرهزینه و سنگین است.
از این رو، برای کاهش پیچیدگی محاسباتی، مدل‌های دیگری نیز معرفی شده‌اند.
به عنوان مثال مدل فیتزهیو-ناگومو\LTRfootnote{FitzHugh-Nagumo} که یک نسخه ساده شده و دو بعدی از مدل هاجکین-هاکسلی است، از جمله این مدل‌هاست
\cite{fitzhugh1961,nagumo1962,dayan2001,graben2008}.

در بسیاری از مسائل فیزیکی، برای کاهش هزینه و افزایش سرعت محاسبات می‌توان از نگاشت‌ها به جای معادلات دیفرانسیل استفاده کرد.
یکی از دلایل این جایگزینی این است که بسیاری از پدیده‌های طبیعی به جزئیات موضعی دینامیک وابسته نیستند، بلکه به برهمکنش تعداد زیادی درجه آزادی غیرخطی وابسته‌اند
\cite{kaneko1992}.
یکی از مدل‌هایی که با این رویکرد برای توصیف نورون ارائه شده، نگاشت چیالوو\LTRfootnote{Chialvo map} است
\cite{chialvo1995,girardi-schappo2013}.

در این پژوهش، تمام مشاهدات، محاسبات و بررسی‌ها تنها بر روی یک مدل تک نورونی که توسط نگاشت چیالوو توصیف می‌شود، انجام خواهد شد.
استفاده از نگاشت چیالوو به دلیل ماهیت معادلات تفاضلی آن، باعث کاهش حجم محاسبات و هزینه‌ها و افزایش سرعت محاسبات می‌شود.
علاوه بر این، این نگاشت به خوبی قادر به نمایش رفتارهای دینامیکی مورد نیاز ما از یک نورون است
\cite{izhikevich2004}.

به دلیل ماهیت احتمالاتی وقوع خطا در یک سیستم، نیاز به سنجه‌ای وجود دارد که تعداد جواب‌های معادله دینامیکی دارای خطا را در فاصله‌ای مشخص از جواب بدون خطا ارزیابی کند.
در این راستا، فیزیک آماری ابزارهای مفیدی را ارائه می‌دهد.
آنتروپی که معیاری احتمالاتی برای اندازه‌گیری اطلاعات در دسترس در یک سیستم است، می‌تواند برای تشخیص خطا در مدل‌های دینامیکی مناسب باشد.
به گونه‌ای که آنتروپی را می‌توان معیاری برای تعداد حالت‌های در دسترس برای یک سیستم دینامیکی که دچار خطایی با مقداری مشخص شده است، در نظر گرفت.
برای محاسبه آنتروپی در هر سیستم، نیاز به تعیین انرژی آزاد آن سیستم داریم که این موضوع نیز در فیزیک آماری به خوبی بررسی شده است.
یکی از روش‌های محاسبه انرژی آزاد، استفاده از تقریب بته و روش انتشار باور است.
همچنین با استفاده از الگوریتم مونته‌کارلو می‌توان به تصحیح خطاهای موجود در آن سیستم دینامیکی پرداخت.

این پایان‌نامه به شرح زیر سازماندهی شده است:
در
\autoref{chap:neuron}،
ساختار نورون و رفتار بیولوژیکی و دینامیکی آن بررسی می‌شود و سپس چند مدل دینامیکی برای نورون معرفی می‌گردد.
در
\autoref{chap:errors}،
روش انتشار باور برای بررسی تعداد جواب‌های دینامیکی در صورت وجود خطا در پارامترهای مدل نگاشت چیالوو معرفی می‌شود و سپس با استفاده از روش مونته‌کارلو، این خطاها در پارامترهای مدل تصحیح می‌گردد.
در نهایت، در
\autoref{chap:discussion}،
نتایج به دست آمده از
\autoref{chap:errors}
جمع‌بندی شده و پیشنهادهایی برای تحقیقات در آینده ارائه می‌شود.
