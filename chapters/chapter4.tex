\chapter{جمع‌بندی و پیشنهادها} \label{chap:discussion}

\section{جمع‌بندی}
نورون‌ها و به تبع آن سیستم عصبی، همواره در معرض بیماری‌ها قرار دارند.
بیماری در یک نورون را می‌توان به صورت اختلالی در پارامترهای مدل دینامیکی آن نورون در نظر گرفت.
این اختلال در پارامترهای مدل دینامیکی منجر به تولید دنباله‌ی دینامیکی متفاوت از دینامیک یک نورون سالم خواهد شد.
برای مشخصه‌یابی اثر این خطاها در مدل، تعداد جواب‌های معادله دینامیکی را که فاصله مشخصی از دنباله‌ی نورون سالم دارند، معین کردیم.
نتایج نشان می‌دهند که با کاهش فاصله از دنباله مربوط به نورون سالم، تعداد جواب‌ها کاهش می‌یابد و این کاهش در مرز ناحیه‌ی برانگیخته نسبت به مرز ناحیه‌ی آشوبناک محسوس‌تر است.
همچنین، در این پژوهش، با استفاده از روش مونته‌کارلو به تصحیح و کاهش خطاهای پارامترها در مدلی از یک نورون پرداختیم.
نتایج نشان می‌دهند که در مرز ناحیه‌ی برانگیخته، با افزایش طول دنباله‌ی مشاهده شده، خطاهای پارامترها به سرعت کاهش یافته و پس از رسیدن به مقداری مشخص، در دنباله‌های بلندتر تقریباً ثابت می‌مانند.
این رفتار یک طول مشخصه را فراهم می‌کند که می‌تواند برای تصحیح خطا بسیار مهم باشد.
به طوری که برای کاهش خطا در مرز ناحیه برانگیخته تنها به دنباله‌هایی با این طول مشخصه نیاز است و نیازی به داشتن دنباله‌های بلندتر ضروری نیست.
در مرز ناحیه‌ی آشوبناک، رفتاری متفاوت مشاهده می‌شود.
به این گونه که با افزایش طول دنباله، خطای پارامترها کاهش می‌یابد.
اما پس از مدتی روندی افزایشی پیدا می‌کند.
علاوه بر این، طول دنباله‌ای که به کمینه خطا منجر می‌شود با مقدار اولیه خطای پارامترها متناسب است، به طوری که خطای اولیه بیشتر، به دنباله‌ای با طول بلندتر برای رسیدن به مقدار کمینه نیاز دارد.

\section{پیشنهادها}
در این پژوهش، به بررسی خطاهای پارامتری در یک مدل دینامیکی از تک نورون پرداختیم.
با این حال، اختلا‌ل‌ها و بیماری‌ها نه تنها در سطح یک نورون، بلکه در ارتباطات بین نورون‌ها و در نهایت در شبکه‌ای از نورون‌ها نیز رخ می‌دهند.
بنابراین، تعمیم این بررسی‌ها به شبکه‌های نورونی می‌تواند گام بعدی این تحقیق باشد.
در این مطالعه، از الگوریتم مونته‌کارلو برای تصحیح خطاها استفاده شد.
به کارگیری الگوریتم‌های بهینه‌تر و دقیق‌تر می‌تواند به عنوان گام بعدی مطرح باشد.
همچنین، در بخش تشخیص خطا، تنها تأثیر شرایط اولیه بر فاصله دنباله‌ها بررسی شد.
در آینده می‌توان تأثیر خطای پارامترها را نیز بر این فاصله مطالعه کرد تا درک عمیق‌تری از تأثیر این خطاها بر رفتار کلی سیستم به دست آید.
علاوه بر این، در این پژوهش، تنها به بررسی رفتار خطاها در مرز بین ناحیه‌ی غیربرانگیخته و برانگیخته و مرز بین ناحیه‌ی غیربرانگیخته و آشوبناک پرداختیم.
در آینده، بررسی رفتار خطا در مرز ناحیه‌ی برانگیخته و آشوبناک که در علوم اعصاب بسیار مهم است نیز باید مدنظر قرار گیرد.
