\thispagestyle{empty}
\addappheadtotoc
\centerline{\textbf{\appendixtocname}}

\begin{appendices}
    \app{انتشار باور} \label{app:belief_propagation}
    در یک سیستم آماری، تعداد حالت‌هایی که انرژی معینی دارند با
    \( \mathcal{N}(E) \)
    نشان داده می‌شود و آنتروپی
    \( S(E) \)
    را بر اساس آن به صورت زیر تعریف می‌شود:
    \begin{equation} \label{eq:number_density}
        \mathcal{N} (E) = e^{N S(E)}
    \end{equation}
    که در آن
    \( E \)
    چگالی انرژی و
    \( S \)
    چگالی آنتروپی است.
    اگر
    \( \Phi(\beta) \)
    را به عنوان چگالی آنتروپی آزاد در نظر بگیریم، می‌توان تابع پارش سیستم را به صورت زیر بازنویسی کرد:
    \begin{equation} \label{eq:partition_function}
        e^{N \Phi(\beta)} = Z = \sum_{\{ s_{i} \}_{i=1}^{N}} e^{-\beta E} = \sum_{E} \sum_{\text{\lr{All states with energy}} E} e^{-N \beta E} = \int dE \, e^{N (S(E) - \beta E)}
    \end{equation}
    در اینجا، جمع روی همه حالت‌ها به جمع روی حالت‌های با انرژی
    \( E \)
    و جمع روی تمام مقادیر مخلتف انرژی تجزیه شده است.
    همچنین، با استفاده از
    \autoref{eq:number_density}
    جمع روی تمام حالت‌ها با انرژی
    \( E \)
    را با تعداد این حالت‌ها جایگزین کرده و سپس جمع روی مقادیر گسسته
    \( E \)
    را به انتگرال تبدیل کرده‌ایم.

    با توجه به
    \autoref{eq:partition_function}
    و در نظر گرفتن حد
    \( N \to \infty \)
    و استفاده از روش نقطه زینی، روابط زیر به دست می‌آیند:
    \begin{equation}
        \left. \frac{\partial S(E)}{\partial E} \right|_{E = E^{*}} = \beta, \qquad \Phi(\beta) = S(E^{*}) - \beta E^{*}
    \end{equation}
    در دمای
    \( \beta \)
    از یک حالت تصادفی
    نمونه‌برداری شود، انرژی آن در
    \( E^{*} = \langle E \rangle \)
    متمرکز است.
    این به این معنی است که توزیع بولتزمن انتخاب مناسبی برای احتمال رخداد یک حالت انرژی معین است، زیرا:
    \begin{equation}
        \frac{d \Phi(\beta)}{d \beta} = - \langle E \rangle
    \end{equation}

    اگر بتوان چگالی آنتروپی آزاد
    \( \Phi \)
    را به عنوان تابعی از
    \( \beta \)
    محاسبه کرد، می‌توان با محاسبه‌ی مشتق‌های آن، میانگین انرژی و تعداد حالت‌ها با یک انرژی معین
    \( S(E) \)
    را به دست آورد.
    این محاسبات معمولاً پیچیده و به لحاظ محاسباتی دشوار است.
    بنابراین، برای به دست آوردن نتایج تقریبی و در برخی موارد دقیق، می‌توان از الگوریتم انتشار باور و آنتروپی آزاد بته استفاده کرد.

    حال تابع توزیع احتمال زیر را در نظر بگیرید:
    \begin{equation} \label{eq:probability_distribution}
        P \left( \{ s_{i} \}_{i=1}^{N} \right) = \frac{1}{Z} \prod_{i=1}^{N} g_{i}(s_{i}) \prod_{a=1}^{M} f_{a}(\{ s_{i} \}_{i \in \partial a})
    \end{equation}
    این تابع توزیع بیان می‌کند که احتمال رخداد حالت
    \( \{ s_{i} \}_{i=1}^{N} \)
    متناسب با احتمال رخداد هر حالت
    \( g_{i}(s_{i}) \)
    و قیدهای
    \( f_{a}(\{ s_{i} \}_{i \in \partial a}) \)
    است که نشان دهنده‌ی برهمکنش‌های سیستم است.
    در اینجا، تعداد متغیرها
    \( N \)
    است و مجموعه‌ای از
    \( M \)
    قید در سیستم وجود دارد.

    همان‌طور که می‌دانیم، کمیت‌های مورد علاقه را می‌توان از ثابت بهنجارش
    \autoref{eq:probability_distribution}
    یا همان تابع پارش استخراج کرد:
    \begin{equation}
        Z = \sum_{\{ s_{i} \}_{i=1}^{N}} \prod_{i}^{N} g_{i}(s_{i}) \prod_{a=1}^{M} f_{a}(\{ s_{i} \}_{i \in \partial a})
    \end{equation}

    حال می‌خواهیم مقدار تابع پارش را برای سیستم محاسبه کنیم.
    برای محاسبه برای این کار، ابتدا دو احتمال زیر را تعریف می‌کنیم:
    \begin{align}
        \chi_{s_{j}}^{j \to a} & = \frac{1}{Z^{j \to a}} g_j(s_j) \prod_{b \in \partial j \backslash a} \psi_{s_{j}}^{b \to j}                                                                                               \\
        \psi_{s_{i}}^{a \to i} & = \frac{1}{Z^{a \to i}} \sum_{\{ s_{j} \}_{j \in \partial a \backslash i}} f_{a} \left( \{ s_{j} \}_{j \in \partial a} \right) \prod_{j \in \partial a \backslash i} \chi_{s_{j}}^{j \to a}
    \end{align}
    که در اینجا،
    \( \chi_{s_{j}}^{j \to a} \)
    احتمال این است که متغیر
    \( j \)
    مقدار
    \( s_{j} \)
    را بگیرد، در حالی که تمام برهمکنش‌های سیستم با متغیر
    \( j \)
    در نظر گرفته شده‌اند و تنها برهمکنش
    \( a \)
    و تمام متغیرهایی که به برهمکنش
    \( a \)
    متصل هستند (به جز
    \( j \))
    از سیستم حذف شده‌اند.
    این احتمال را می‌توان به شکل پیامی از متغیر
    \( j \)
    به برهمکنش
    \( a \)
    تعبیر کرد.
    همچنین،
    \( \psi_{s_{i}}^{a \to i} \)
    احتمال این است که متغیر
    \( i \)
    مقدار
    \( s_{i} \)
    را بگیرد، در صورتی که تمام برهمکنش های متصل به متغیر
    \( i \)
    به جز برهمکنش
    \( a \)
    حذف شده باشند.
    این احتمال را می‌توان به شکل پیامی از برهمکنش
    \( a \)
    به متغیر
    \( i \)
    تعبیر کرد.

    ثابت‌های بهنجارش
    \( Z^{j \to a} \) و \( Z^{a \to i} \)
    که در معادله‌های بالا ظاهر شدند، به صورت زیر نوشته می‌شوند:
    \begin{align}
        Z^{j \to a} & = \sum_{s_i} g_j(s_j) \prod_{b \in \partial j \backslash a} \psi_{s_{j}}^{b \to j} \\
        \begin{split}
            Z^{a \to i} & = \sum_{s_{i}} \sum_{\{ s_{j} \}_{j \in \partial a \backslash i}} f_{a} \left( \{ s_{j} \}_{j \in \partial a} \right) \prod_{j \in \partial a \backslash i} \chi_{s_{j}}^{j \to a} \\
                        & = \sum_{\{ s_{j} \}_{j \in \partial a}} f_{a} \left( \{ s_{j} \}_{j \in \partial a} \right) \prod_{j \in \partial a \backslash i} \chi_{s_{j}}^{j \to a}
        \end{split}
    \end{align}

    این معادله‌ها که معادله‌هایی خودسازگار هستند، به عنوان معادله‌های انتشار باور شناخته می‌شوند.

    جواب معادله‌های انتشار باور، مجموعه‌ای از احتمال‌های حاشیه‌ای
    \( \chi^{i \to j} \)
    را فراهم می‌کند که می‌توان از آن‌ها برای محاسبه آنتروپی آزاد بته
    \( \Phi \)
    استفاده کرد.
    این احتمال‌های حاشیه‌ای تخمین‌هایی از احتمال‌های حاشیه‌ای واقعی متغیرها هستند و همچنین آنتروپی آزاد بته تخمینی برای آنتروپی آزاد واقعی است.

    بنابراین، با یافتن جواب معادله‌های انتشار باور، می‌توان آنتروپی آزاد و میانگین انرژی را به صورت تقریبی محاسبه کرد.
    با داشتن آنتروپی آزاد
    \( \Phi \)
    و میانگین انرژی
    \( E^{*} \)
    می‌توان آنتروپی
    \( S(E) \)
    را به راحتی محاسبه کرد:
    \[ S(E^{*}) = \Phi(\beta) + \beta E^{*} \]

    به طور خاص، مقدار آنتروپی آزاد
    \( N \Phi = \ln Z \)
    و توزیع احتمال‌های حاشیه‌ای برای هر متغیر، کمیت‌های مورد علاقه در هر سیستم هستند:
    \[ \mu_{i}(s_{i}) = \sum_{\{ s_{j} \}_{j=1}^{N}, j \ne i} P \left( \{ s_{j} \}_{j=1}^{N} \right) \]

    معادله‌های انتشار باور که پیش‌تر معرفی شدند، در قالب اصلی خود برای محاسبه‌ی تابع پارش سیستم مناسب نیستد، زیرا این معادله‌ها به انتخاب ریشه‌ی درخت شبکه‌ی برهمکنش‌ها وابسته‌اند.
    بنابراین، برای محاسبه تابع پارش، به معادله‌ای نیاز داریم که هم شامل پیام‌های
    \( \chi \) و \( \psi \)
    باشد و هم به گونه‌ای مستقل از انتخاب ریشه‌ی شبکه‌ی برهمکنش تعریف شده باشد.
    به همین منظور، معادله‌های زیر معرفی شده‌اند:
    \begin{align}
        \label{eq:z_variable}
        Z^{i}  & = \sum_{r} g_{i}(s_{i}) \prod_{a \in \partial i} \psi_{r}^{a \to i}                                                           \\
        Z^{a}  & = \sum_{\{ s_{i} \}_{i \in \partial a}} f_{a}(\{ s_{i} \}_{i \in \partial a}) \prod_{i \in \partial a} \chi_{s_{i}}^{i \to a} \\
        Z^{ia} & = \sum_{r} \chi_{r}^{i \to a} \psi_{r}^{a \to i}
    \end{align}

    تابع پارش را می‌توان با استفاده از این معادله‌ها به شکل زیر بیان کرد:
    \begin{equation}
        Z = \frac{\prod_{i=1}^{N} Z^{i} \prod_{a=1}^{M} Z^{a}}{\prod_{(ia) \in E} Z^{ia}}
    \end{equation}

    تفسیر این معادله‌ها به شرح زیر است:
    \begin{itemize}[label=-]
        \item \( Z^{i} \)
              نشان دهنده‌ی تغییر در تابع پارش
              \( Z \)
              هنگام اضافه شدن متغیر
              \( i \)
              به سیستم است.
        \item \( Z^{a} \)
              بیانگر تغییر در تابع پارش
              \( Z \)
              است وقتی برهمکنش
              \( a \)
              به سیستم اضافه شود.
        \item \( Z^{ia} \)
              نشان دهنده‌ی تغییر در تابع پارش
              \( Z \)
              است وقتی متغیر
              \( i \)
              و برهمکنش
              \( a \)
              به هم متصل می‌شوند.
    \end{itemize}

    بنابراین، تابع پارش را می‌توان به صورت زیر توصیف کرد.
    ابتدا، همه‌ی مشارکت‌های متغیرها و برهمکنش‌ها در تابع پارش در نظر گرفته می‌شوند.
    با اضافه کردن هر متغیر یا برهمکنش جدید، تمامی اتصال‌های مرتبط به آن نیز لحاظ می‌شوند،
    به همین دلیل، هر اتصال میان متغیرها و برهمکنش‌ها دقیقاً دو بار شمارش می‌شود.
    در نتیجه، برای تصحیح این شمارش مضاعف، باید سهم اتصال‌ها را به اندازه‌ی یک بار از کل تابع پارش کم کنیم.

    برای محاسبه‌ی احتمال‌های حاشیه‌ای
    \( \mu_{i}(s_{i}) \)
    از معادله‌ی زیر استفاده می‌شود:
    \begin{equation}
        \mu_{i}(s_{i}) = \frac{1}{Z^{i}} g_{i}(s_{i}) \prod_{a \in \partial i} \psi_{s_{i}}^{a \to i}
    \end{equation}
    که در آن ثابت بهنجارش
    \( Z^{i} \)
    توسط
    \autoref{eq:z_variable}
    تعریف شده است.
    همان طور که مشاهده می‌کنید،
    هر احتمال حاشیه‌ای
    \( \mu_{i}(s_{i}) \)
    یک تابع ساده از پیام‌های دریافتی
    \( \psi_{s_{i}}^{a \to i} \)
    از تمام برهمکنش‌های
    \( a \)
    متصل به متغیر
    \( i \)
    است.

    در سیستم‌هایی که برهمکنش‌ها دوتایی هستند، معادله‌های انتشار باور را می‌توان به شکل ساده‌تری نوشت.
    برای دو متغیر همسایه
    \( i \) و \( j \)،
    تابع پارش جزئی
    \( Z_{i \to j}(s_i) \)
    را در نظر بگیرید.
    این عبارت نشان دهنده‌ی تابع پارش متغیر
    \( i \)
    است با فرض این که مقدار
    \( s_{i} \)
    را دارد و اتصال آن با متغیر
    \( j \)
    نیز نادیده گرفته شده است.
    به همین ترتیب،
    \( Z_{i}(s_{i}) \)
    تابع پارش جزئی متغیر
    \( i \)
    است وقتی مقدار
    \( s_{i} \)
    را دارد.
    بنابراین، معادله‌های انتشار باور را می‌توان به شکل زیر نوشت:
    \begin{align}
        \label{eq:partial_partition_function_constraint}
        Z_{i \to j}(s_{i}) & = \psi_{i}(s_{i}) \prod_{k \in \partial i \backslash j} \left( \sum_{s_{k}} Z_{k \to i}(s_{k}) \psi_{ik}(s_{i},s_{k}) \right) \\
        \label{eq:partial_partition_function_variable}
        Z_{i}(s_{i})       & = \psi_{i}(s_{i}) \prod_{j \in \partial i} \left( \sum_{s_{j}} Z_{j \to i}(s_{j}) \psi_{ij}(s_{i},s_{j}) \right)
    \end{align}

    کار با تابع پارش در بسیاری از موارد می‌تواند دشوار و حتی گاهی غیرممکن باشد.
    بنابراین، می‌توان با بهنجار کردن معادله‌های
    \ref{eq:partial_partition_function_constraint} و \ref{eq:partial_partition_function_variable}،
    آن‌ها را به شکل احتمال بازنویسی کرد:
    \begin{align}
        \label{eq:marginal_constraint}
        \eta_{i \to j}(s_{i}) & = \frac{Z_{i \to j}(s_{j})}{\sum_{s'_{i}} Z_{i \to j}(s'_{j})} \\
        \label{eq:marginal_variable}
        \eta_{i}(s_{i})       & = \frac{Z_{i}(s_{i})}{\sum_{s'_{i}} Z_{i}(s'_{i})}
    \end{align}
    که در اینجا،
    \( \eta_{i \to j}(s_{i}) \)
    توزیع احتمال حاشیه‌ای متغیر
    \( s_{i} \)
    است هنگامی که برهمکنش
    \( (ij) \)
    حذف شده است و
    \( \eta_{i}(s_{i}) \)
    توزیع احتمال حاشیه‌ای متغیر
    \( s_{i} \)
    است هنگامی که همه‌ی برهمکنش‌ها در نظر گرفته شده‌اند.

    با استفاده از معادله‌های
    \ref{eq:partial_partition_function_constraint} و \ref{eq:partial_partition_function_variable}
    می‌توان معادله‌های
    \ref{eq:marginal_constraint} و \ref{eq:marginal_variable}
    به شکل زیر بازنویسی کرد:
    \begin{align}
        \label{eq:probability_constraint}
        \eta_{i \to j}(s_{i}) & = \frac{\psi_{i}(s_{i})}{z_{i \to j}} \prod_{k \in \partial i \backslash j} \left( \sum_{s_{k}} \eta_{k \to i} (s_k) \psi_{ik}(s_{i},s_{k}) \right) \\
        \label{eq:probability_variable}
        \eta_{i}(s_{i})       & = \frac{\psi_{i}(s_{i})}{z_{i}} \prod_{j \in \partial i} \left( \sum_{s_{j}} \eta_{j \to i} (s_{j}) \psi_{ij}(s_{i},s_{j}) \right)
    \end{align}
    که در اینجا
    \( z_{i \to j} \) و \( z_{i} \)
    ثابت‌های بهنجارش هستند که به شکل زیر تعریف می‌شوند:
    \begin{align}
        \label{eq:normalization_constraint}
        z_{i \to j} & = \sum_{s_{i}} \psi_{i}(s_{i}) \prod_{k \in \partial i \backslash j} \left( \sum_{s_{k}} \eta_{k \to i}(s_{k}) \psi_{ik}(s_{i},s_{k}) \right) \\
        \label{eq:normalization_variable}
        z_{i}       & = \sum_{s_{i}} \psi_{i}(s_{i}) \prod_{j \in \partial i} \left( \sum_{s_{j}} \eta_{j \to i}(s_{j}) \psi_{ij}(s_{i},s_{j}) \right)
    \end{align}

    معادله‌های
    \ref{eq:probability_constraint} و \ref{eq:probability_variable}
    به همراه معادله‌های بهنجارش
    \ref{eq:normalization_constraint} و \ref{eq:normalization_variable}
    معادله‌های انتشار باور را تشکیل می‌دهند.
    به احتمال حاشیه‌ای
    \( \eta_{i \to j}(s_{i}) \)،
    احتمال حفره نیز گفته می‌شود زیرا با حذف یک متغیر از شبکه برهمکنش‌ها به دست آمده است.

    برای محاسبه‌ی تابع پارش، ابتدا کمیتی به صورت زیر برای هر برهمکنش
    \( (ij) \)
    تعریف می‌شود:
    \begin{equation} \label{eq:z_constraint}
        z_{ij} = \sum_{s_{i},s_{j}} \eta_{j \to i}(s_{j}) \eta_{i \to j}(s_{i}) \psi_{ij}(s_{i}, s_{j}) = \frac{z_{j}}{z_{j \to i}} = \frac{z_{i}}{z_{i \to j}}
    \end{equation}
    سپس، با استفاده از معادله‌های
    \ref{eq:normalization_variable}، \ref{eq:marginal_constraint} و \ref{eq:partial_partition_function_variable}،
    تابع پارش جزئی متغیر
    \( i \)
    را می‌توان به شکل زیر نوشت:
    \begin{equation}
        \begin{split}
            z_{i} & = \sum_{s_{i}} \psi_{i}(s_{i}) \prod_{j \in \partial i} \left( \sum_{s_{j}} \eta_{j \to i} (s_{j}) \psi_{ij}(s_{i},s_{j}) \right)                                       \\
                  & = \sum_{s_{i}} \psi_{i}(s_{i}) \prod_{j \in \partial i} \left( \sum_{s_{j}} \frac{Z_{j \to i}(s_{j})}{\sum_{s'_{j}} Z_{j \to i}(s'_{j})} \psi_{ij}(s_{i},s_{j}) \right) \\
                  & = \frac{\sum_{s_{i}} Z_{i}(s_{i})}{\prod_{j \in \partial i } \sum_{s_{j}} Z_{j \to i}(s_{j})}
        \end{split}
    \end{equation}

    به طور مشابه می‌توان نوشت:
    \begin{equation}
        z_{j \to i} = \frac{\sum_{s_{j}} Z_{j \to i}(s_{j})}{\prod_{k \in \partial j \backslash i} \sum_{s_{k}} Z_{k \to j}(s_{k})}
    \end{equation}

    برای همه‌ی حالت‌های
    \( s_{i} \)،
    می‌توان تابع پارش کل را با استفاده از رابطه زیر به دست آورد:
    \[ Z = \sum_{s_i} Z_i(s_i) \]

    بنابراین، می‌توان از یک متغیر دلخواه
    \( i \)
    شروع کرده و رابطه زیر را برای تابع پارش نوشت:
    \[ Z = \sum_{s_{i}} Z_{i}(s_{i}) = z_{i} \prod_{j \in \partial i} \left( \sum_{s_{j}} Z_{j \to i}(s_{j}) \right) = z_i \prod_{j \in \partial i} \left( z_{j \to i} \prod_{k \in \partial j \backslash i} \sum_{s_{k}} Z_{k \to j}(s_{k}) \right) \]

    با استفاده از
    \autoref{eq:z_constraint}
    می‌توان تمام احتمال‌های حاشیه‌ای
    \( z_{j \to i} \)
    را که در معادله بالا ظاهر شده‌اند با احتمال‌های حاشیه‌ای متغیرها و برهمکنش‌ها جایگزین کرد.
    به این ترتیب، به رابطه زیر خواهیم رسید:
    \[ Z = z_{i} \prod_{j \in \partial i} \left( z_{j \to i} \prod_{k \in \partial j \backslash i} z_{k \to j} \cdots \right) = z_{i} \prod_{j \in \partial i} \left( \frac{z_{j}}{z_{ij}} \prod_{k \in \partial j \backslash i} \frac{z_{k}}{z_{jk}} \cdots \right) = \frac{\prod_{i} z_{i}}{\prod_{(ij)} z_{ij}} \]
    این شکل از تابع پارش، محاسبه‌ی آنتروپی آزاد را به گونه‌ای ممکن می‌سازد که بتوان مستقیماً از پیام‌های حفره برای انجام این محاسبه استفاده کرد.
    به این آنتروپی آزاد، آنتروپی آزاد بته گفته می‌شود و به صورت زیر محاسبه می‌شود:
    \begin{equation}
        N \Phi = \ln Z = \sum_{i} f_{i} - \sum_{(ij)} f_{ij}
    \end{equation}
    که در اینجا
    \( f_{i} = \ln z_{i} \) و \( f_{ij} = \ln z_{ij} \)
    هستند.
    \( f_{i} \)
    از بهنجار کردن توزیع احتمال حاشیه‌ای متغیر
    \( i \)
    به دست آمده است و تغییر در تابع پارش را هنگام افزودن متغیر
    \( i \)
    و برهمکنش‌های مرتبط با آن به سیستم نشان می‌دهد.
    همچنین،
    \( f_{ij} \)
    تغییر در تابع پارش را هنگام افزودن برهمکنش
    \( (ij) \)
    به سیستم نشان می‌دهد.
    این معادله، تفسیری ساده از آنتروپی آزاد بته ارائه می‌دهد.
    آنتروپی آزاد کل سیستم برابر است با مجموع آنتروپی آزاد
    \( f_{i} \)
    برای همه‌ی متغیر‌ها و برهمکنش‌های آن‌ها، اما از آنجا که هر برهمکنش در این مجموع دو بار شمارش شده است، برای تصحیح این شمارش اضافه،
    \( f_{ij} \)
    را از مجموع کم می‌کنیم.

    \app{گزیده‌ای از تابع‌های استفاده شده در شبیه‌سازی‌ها}
    تمام شبیه‌سازی‌های این پژوهش با استفاده از زبان برنامه نویسی راست\LTRfootnote{Rust programming language} انجام شده است.
    تمامی کدهای نوشته شده در
    \cite{sarikhani2024}
    ارائه شده‌اند.
    در این پیوست تنها برخی از تابع‌های مهم آورده شده است.

    \vspace{\baselineskip}
    تابع‌های الگوریتم انتشار باور و دینامیک جمعیت برای محاسبه تعداد جواب‌های دینامیکی
    \begin{latin}
        \inputminted[firstline=155,lastline=178]{rust}{../codes/bpmc/src/bin/bp.rs}
        \inputminted[firstline=206,lastline=262]{rust}{../codes/bpmc/src/bin/bp.rs}
        \inputminted[firstline=264,lastline=275]{rust}{../codes/bpmc/src/bin/bp.rs}
        \inputminted[firstline=277,lastline=298]{rust}{../codes/bpmc/src/bin/bp.rs}
    \end{latin}

    \clearpage

    تابع‌های الگوریتم مونته‌کارلو برای تصحیح خطا در پارامترها
    \begin{latin}
        \inputminted[firstline=57,lastline=83]{rust}{../codes/bpmc/src/bin/mc.rs}
        \inputminted[firstline=85,lastline=104]{rust}{../codes/bpmc/src/bin/mc.rs}
    \end{latin}
\end{appendices}
